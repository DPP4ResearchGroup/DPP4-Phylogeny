The workflow to produce the phylogenetic analysis for DPP4 across species has been outlined in Figure \ref{fig:pipeline}. By limiting the organism, the homologues protein sequence in a particular specie can be retrieved by blasting against reference human DPP4 using protein using BLASTP. Given the high similarity between DPP4 family proteases (DPP4, FAP, DPP8 and DPP9), retrieved sequences from previous blasting need to be validated against DPP4 motif and subjected to size test (in the size range roughly between 750 and 768 aa) before been selected to include in further analysis. By repeating the process, we can obtain a collection of DPP4 evolution profile across targeted taxonomy. The collected sequences are required to be formatted in a neat fasta format before can be multi-sequence aligned by MUSCLE~\cite{15318951}. Although there are a few multi-alignment tools available, MUSCLE has been used for this analysis pipeline, the selection was based on the the speed and the accuracy. With the multi-alignment result, we can then analysis the DPP4 evolution by joint neighbour using PHYLIP. \\

The rapid development in genomic sequencing has allowed many species to be whole genome sequenced for the first time. Public sequence repositories like UniProtKB and NCBI protein database have rapidly increase in volume from autonomous annotation on newly available whole genome sequences. This in turn has now provided us an opportunity to study the evolution of species at a protein level.  Due to the limitation of incomplete annotation of proteome across many species, the evolution study at proteomics level across whole kingdom is still limited. 