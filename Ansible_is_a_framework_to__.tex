Ansible\textsuperscript{\textcopyright} is a framework to support and automate the usually tedious and laborious software deployment processes. With ansible framework, it drastically improve the reproducibility, deployability and scalability of a service. Therefore, 
an ansible role for Jupyter notebook has been developed to autonomously initiate a machine with Linux Ubuntu 16.04 operating system. 
Deploy a application involves many error prone steps and application is extremely sensitive to environment settings and association status, this is particular true for deploying a service application. To test the whole application each step of the way along with its deployment environment is commonly referred as continous testing and is arguably the most effective development strategy to date. In order to create a robust and reproducible deployment strategy, continuous test mechanism has been deployed to autonomously check each deployment cycle and ensure the robustness of the mechanism. 

To establish such a system (as in Figure~\ref{fig:infrastructure}), we setup two independent repo with different purposes and are identified as Ansible Playbook and Notebook.  Ansible Playbook Repo contains ansible\textsuperscript{\textcopyright} playbook and configuration files, which is automatically pulled by travis engine and used to configure the hosting environment (i.e. Jupyter\textsuperscript{\textcopyright} server) from vanilla state. On the other hand, Notebook repo contains analysis and algorithm pipelines in Jupyter~\textsuperscript{\textcopyright} Notebook format, which is automatically triggered the deployment/redeployment mechanism upon new commits (changes) are detected in the repository. Before a new version of analysis pipeline can be deployed on the server, there is a series of dedicated sanity checks (i.e. Pytest) needs to be passed to ensure a stable and clean deployment. Once both Jupyter\textsuperscript{\textcopyright} server and analysis pipeline are successfully deployed, Jupyter\textsuperscript{\textcopyright} server will be automatically started/restarted and the access to the interactive analysis pipeline will be available through dedicated port or a secure shell tunnelling via TCP protocol (e.g. access via a common web browser). 