The dipeptidyl peptidase-4 (DPP4) is a 766 amino acid protein with exo-hydrolytic activity in human. DPP4 has been reported to exist in  monomer, homo-dimmer and homo-tetramer forms~\textit{in vivo} with homo-dimmer represents the major catalytic active form.~\cite{Mulvihill_2014} Ubiquitous expression and enzymatic activity of DPP4 in human has been reported in epithelial cells of intestine, kidney, spleen, lung, thymus and lymph nodes. 

Molecularlly, the catalytic region of DPP4 contains a catalytic triad (Ser630, Asp708, His740) with an active Serine site at Ser630. Monomeric DPP4 has a short N-terminal cytoplasmic portion consists of first 6 residues, follows by a transmembrane range that spans 22 aa residues before the rest structures flip to extracellular space, which comprises of 8 blades of \beta-propeller and a large \alpha/\beta-hydrolase domain. The C-terminal \alpha/\beta-hydrolase domain remains consistent in large across species, while N-terminal 8 blades \beta-propeller regions evolve constantly. 

Apart from enzymatic functions, DPP4 is also recognised as a cell-surface receptor, which is commonly as CD-26. Structural variations at extracellular domains alters signalling functions across species. Recent study in bats~\cite{Letko_2018, Cui_2013} has suggested that the action of the selection pressure on 8 blade \beta-propeller region contribute to the differentiated reception in viral recognition.  

\\
DPP4 functions in human 
\\ 
In relation to other animals 

