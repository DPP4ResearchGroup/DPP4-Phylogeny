In this study, we investigated the evolution of DPP4 protease across species within super class Osteichthyes under clade Craniata/Vertebrata, phylum Chordate using the pipeline we developed earlier (as shown in Figure~\ref{fig:pipeline}). 

\begin{table}[h!]
\label{tab:Tax} 
 \begin{center}
 \caption{Taxonomy of Osteichthyes.~\cite{18563158}} 
    \begin{tabular}{| c | c |}
    \hline
        Kindom & Animalia \\ 
        Phylum & Chordata \\ 
        Clade & Olfactores \\ 
        Clade & Craniata \\ 
        Superclass &  Osteichthyes\\ 
    \hline
    \end{tabular}
 \end{center}
\end{table}

Non-redundant protein sequence collection has been exclusively used to retrieve DPP4 protein sequences from, which consists protein sequences recorded in databases from GeneBank CDS (coding sequence) translations, PDB, SwissProt, PIR (Protein Information Resource), PRF (Programmed Ribosomal Frameshift). 